% !TEX root = ./notes_template.tex
%%%%%%%%%%%%%%%%%%%%%%%%%%%%%%%%%%%%%%%%%%%%%%%%%%
%%%%%%%%%%%%%%%%%%%%% preamble %%%%%%%%%%%%%%%%%%%
%%%%%%%%%%%%%%%%%%%%%%%%%%%%%%%%%%%%%%%%%%%%%%%%%%
\documentclass[11pt,twoside]{book}

% \usepackage[mono=false]{libertine} % new linux font, ignore mono

\usepackage{luatex85}

\renewcommand{\baselinestretch}{1.05}
\usepackage{amsmath,amsthm,amssymb,mathrsfs,amsfonts,dsfont}
\usepackage{epsfig,graphicx}
\usepackage{tikz}
\usepackage{tikz-3dplot} % Required for 3D plots
\usepackage{tabularx}
\usepackage{blkarray}
\usepackage{slashed}
\usepackage{color}
\usepackage{listings}
\usepackage{caption}
% \usepackage{fullpage}
\usepackage{lipsum} % provides dummy text for testing
\usepackage[toc,title,titletoc,header]{appendix}
\usepackage{minitoc}
\usepackage{color}
\usepackage{multicol} % two-col ToC
\usepackage{bm}
\usepackage{imakeidx} % before hyperref
\usepackage{hyperref}
% link colors settings
\hypersetup{
    colorlinks=true,
    citecolor=magenta,
    linkcolor=blue,
    filecolor=green,      
    urlcolor=cyan,
    % hypertexnames=false,
}
\usepackage[capitalise]{cleveref}
\usepackage{subcaption}
\usepackage{enumitem}
\usepackage{mathtools}
\usepackage{physics}
\usepackage[linesnumbered,ruled,vlined,algosection]{algorithm2e}
\SetCommentSty{textsf}
\usepackage{epigraph}
\epigraphwidth=1.0\linewidth
\epigraphrule=0pt

% adjust margin
\usepackage[margin=2.3cm]{geometry}
\headheight13.6pt

%%%%%%%%%%%%%%%% thmtools %%%%%%%%%%%%%%%%%%%%%
\usepackage{thmtools}
\declaretheorem[numberwithin=chapter]{theorem}
\declaretheorem[numberwithin=chapter]{axiom}
\declaretheorem[numberwithin=chapter]{lemma}
\declaretheorem[numberwithin=chapter]{proposition}
\declaretheorem[numberwithin=chapter]{claim}
\declaretheorem[numberwithin=chapter]{conjecture}
\declaretheorem[sibling=theorem]{corollary}
\declaretheorem[numberwithin=chapter, style=definition]{definition}
\declaretheorem[numberwithin=chapter, style=definition]{problem}
\declaretheorem[numberwithin=chapter, style=definition]{example}
\declaretheorem[numberwithin=chapter, style=definition]{exercise}
\declaretheorem[numberwithin=chapter, style=definition]{observation}
\declaretheorem[numberwithin=chapter, style=definition]{fact}
\declaretheorem[numberwithin=chapter, style=definition]{construction}
\declaretheorem[numberwithin=chapter, style=definition]{remark}
\declaretheorem[numberwithin=chapter, style=remark]{question}
%%%%%%%%%%%%%%%% thmtools %%%%%%%%%%%%%%%%%%%%%
\usepackage{changepage}
\newenvironment{solution}
    {\renewcommand\qedsymbol{$\square$}\color{blue}\begin{adjustwidth}{0em}{2em}\begin{proof}[\textit Solution.~]}
    {\end{proof}\end{adjustwidth}}

%%%%%%%%%%%%%%%% index %%%%%%%%%%%%%%%%%%%%%
\begin{filecontents}{index.ist}
% https://tex.stackexchange.com/questions/65247/index-with-an-initial-letter-of-the-group
headings_flag 1
heading_prefix "{\\centering\\large \\textbf{"
heading_suffix "}}\\nopagebreak\n"
delim_0 "\\nobreak\\dotfill"
\end{filecontents}
\newcommand{\myindex}[1]{\index{#1} \emph{#1}}
\makeindex[columns=3, intoc, title=Alphabetical Index, options= -s index.ist]
%%%%%%%%%%%%%%%% index %%%%%%%%%%%%%%%%%%%%%

%%%%%%%%%%%%%%%% ToC %%%%%%%%%%%%%%%%%%%%%
% Link Chapter title to ToC: https://tex.stackexchange.com/questions/32495/linking-the-section-text-to-the-toc
\usepackage[explicit]{titlesec}
\titleformat{\chapter}[display]
  {\normalfont\huge\bfseries}{\chaptertitlename\ {\thechapter}}{20pt}{\hyperlink{chap-\thechapter}{\Huge#1}
\addtocontents{toc}{\protect\hypertarget{chap-\thechapter}{}}}
\titleformat{name=\chapter,numberless}
  {\normalfont\huge\bfseries}{}{-20pt}{\Huge#1}

%%%%%%%%%%%%%%%%%%% fancyhdr %%%%%%%%%%%%%%%%%
\usepackage{fancyhdr}
\pagestyle{fancy} % enable fancy page style
\renewcommand{\headrulewidth}{0.0pt} % comment if you want the rule
\fancyhf{} % clear header and footer
\fancyhead[lo,le]{\leftmark}
\fancyhead[re,ro]{\rightmark}
\fancyfoot[CE,CO]{\hyperref[toc-contents]{\thepage}}

% https://tex.stackexchange.com/questions/550520/making-each-page-number-link-back-to-beginning-of-chapter-or-section
\makeatletter
\def\chaptermark#1{\markboth{\protect\hyper@linkstart{link}{\@currentHref}{Chapter \thechapter ~ #1}\protect\hyper@linkend}{}}
\def\sectionmark#1{\markright{\protect\hyper@linkstart{link}{\@currentHref}{\thesection ~ #1}\protect\hyper@linkend}}
\makeatother
%%%%%%%%%%%%%%%%%%% fancyhdr %%%%%%%%%%%%%%%%%


%%%%%%%%%%%%%%%%%%% biblatex %%%%%%%%%%%%%%%%%
\usepackage[doi=false,url=false,isbn=false,style=alphabetic,backend=biber,backref=true]{biblatex}
\addbibresource{bib.bib}

\newbibmacro{string+doiurlisbn}[1]{%
  \iffieldundef{doi}{%
    \iffieldundef{url}{%
      \iffieldundef{isbn}{%
        \iffieldundef{issn}{%
          #1%
        }{%
          \href{http://books.google.com/books?vid=ISSN\thefield{issn}}{#1}%
        }%
      }{%
        \href{http://books.google.com/books?vid=ISBN\thefield{isbn}}{#1}%
      }%
    }{%
      \href{\thefield{url}}{#1}%
    }%
  }{%
    \href{http://dx.doi.org/\thefield{doi}}{#1}%
  }%
}

% https://tex.stackexchange.com/questions/94089/remove-quotes-from-inbook-reference-title-with-biblatex
\DeclareFieldFormat[article,incollection,inproceedings,book,misc]{title}{\usebibmacro{string+doiurlisbn}{\mkbibemph{#1}}}
% https://tex.stackexchange.com/questions/454672/biblatex-journal-name-non-italic
\DeclareFieldFormat{journaltitle}{#1\isdot}
\DeclareFieldFormat{booktitle}{#1\isdot}
% https://tex.stackexchange.com/questions/10682/suppress-in-biblatex
\renewbibmacro{in:}{}
% add video field: https://tex.stackexchange.com/questions/111846/biblatex-2-custom-fields-only-one-is-working
\DeclareSourcemap{
    \maps[datatype=bibtex]{
      \map{
        \step[fieldsource=video]
        \step[fieldset=usera,origfieldval]
    }
  }
}
\DeclareFieldFormat{usera}{\href{#1}{\textsc{Online video}}}
\AtEveryBibitem{
    \csappto{blx@bbx@\thefield{entrytype}}{% put at end of entry
        \iffieldundef{usera}{}{\space \printfield{usera}}
    }
}
%%%%%%%%%%%%%%%%%%% biblatex %%%%%%%%%%%%%%%%%

%%%%%%%%%%%%%%%%%%%%% glossaries %%%%%%%%%%%%%%%%%
% !TEX root = ./notes_template.tex
% \usepackage[style=super]{glossaries}
% https://www.overleaf.com/learn/latex/Glossaries
\usepackage[style=super,toc,acronym]{glossaries}
\setlength{\glsdescwidth}{1\linewidth}
\makeglossaries

\renewcommand\glossaryname{List of Abbreviations and Symbols}

\newglossaryentry{Q2}{name={$Q_2(f)$},
%sort=Q2,
description={Two-side (bounded) error quantum query complexity}}

\newglossaryentry{real_number}{name={$\mathbb{R}$},description={Real number}}

% \newglossaryentry{gcd}{name={gcd},description={greatest common divisor}}

\newacronym{gcd}{GCD}{Greatest Common Divisor}


\newglossaryentry{svm}{name={SVM},description={Support Vector Machine}}

\newglossaryentry{gd}{name={GD},description={Gradient Descent}}

\newglossaryentry{qft}{name={QFT},description={Quantum Field Theory}}

\newglossaryentry{qm}{name={QM},description={Quantum Mechanics}}

\newglossaryentry{v}{name={$\vec{v}$},description={a vector}}

% physics
\newglossaryentry{hamiltonian}{name={$\hat{H}$},description={Hamiltonian}}

\newglossaryentry{lagrangian}{name={$L$},description={Lagrangian}}
%%%%%%%%%%%%%%%%%%%%% glossaries %%%%%%%%%%%%%%%%%

%%%%%%%%%%%%%%%%%%%%% glossaries-extra %%%%%%%%%%%%%%%%%
% \usepackage[record,abbreviations,symbols,stylemods={list,tree,mcols}]{glossaries-extra}
%%%%%%%%%%%%%%%%%%%%% glossaries-extra %%%%%%%%%%%%%%%%%

% !TEX root = ./notes_template.tex

%%%%%%%%%%%%%%%%%%%%%%%%%%%%%%%%%%%%
%%%%%%%%%%%%%%%%%%%%%%%%%%%%%%%%%%%%
% math
\let\iff\relax
\newcommand{\iff}{\text{ iff }}
\newcommand{\OPT}{\textup{OPT}}

% physics
\newcommand{\acreation}{a^\dagger}



%%%%%%%%%%%%%%%%%%%%% Custom-Defintions %%%%%%%%%%%%%%%%%
\def\mf{\ensuremath\mathbf}
\def\mb{\ensuremath\mathbb}
\def\mc{\ensuremath\mathcal}
\def\lp{\ensuremath\left(}
\def\rp{\ensuremath\right)}
\def\lv{\ensuremath\left\lvert}
\def\rv{\ensuremath\right\rvert}
\def\lV{\ensuremath\left\lVert}
\def\rV{\ensuremath\right\rVert}
\def\lc{\ensuremath\left\{}
\def\rc{\ensuremath\right\}}
\def\ls{\ensuremath\left[}
\def\rs{\ensuremath\right]}
\def\bmx{\ensuremath\begin{bmatrix*}[r]}
\def\emx{\ensuremath\end{bmatrix*}}
\def\bmxc{\ensuremath\begin{bmatrix*}[c]}

\newcommand{\demoex}[2]{\onslide<#1->\begin{color}{black!60} #2 \end{color}}
\newcommand{\demoexc}[3]{\onslide<#1->\begin{color}{#2} #3 \end{color}}
\newcommand{\anim}[3]{\onslide<#1->{\begin{color}{#2!60} #3 \end{color}}}
\newcommand{\ct}[1]{\lp #1\rp}
\newcommand{\dt}[1]{\ls #1\rs}
\newcommand{\cols}[2]{\begin{columns}[#1] #2 \end{columns}}
\newcommand{\col}[2]{\begin{column}{#1} #2 \end{column}}
\newcommand{\eqnwl}[1]{\begin{equation} #1 \end{equation}}
\newcommand{\eqnwol}[1]{\begin{equation*} #1 \end{equation*}}

%%%%%%%%%%%%%%%%%%%%%%%%%%%%%%%%%%%%%%%%%%%%%%%%%%
%%%%%%%%%%%%%%%% begin of document %%%%%%%%%%%%%%%
%%%%%%%%%%%%%%%%%%%%%%%%%%%%%%%%%%%%%%%%%%%%%%%%%%

\begin{document}

\title{\bf \huge Mechatronics for Rehabilitation Enginbeering: Course Notes}
\author{Sivakumar Balasubramanian \\ CMC Vellore}
\date{Update on \today}
\maketitle
\setcounter{tocdepth}{2}
\setcounter{minitocdepth}{1} 

\begin{multicols}{2}
    \dominitoc% Initialization
    \adjustmtc[2]% chp number shift for mini-toc
    \tableofcontents
    \label{toc-contents}
\end{multicols}

% 	\listoffigures
% 	% \listoftables
% \begin{multicols}{2}
% 	\listoftheorems[ignoreall,show={theorem}]
% \end{multicols}

% 	\renewcommand{\listtheoremname}{List of Definitions}
% \begin{multicols}{2}
% 	\listoftheorems[ignoreall,show={definition}]
% \end{multicols}

	% \printglossaries
	% \printglossary[type=\acronymtype]
	% \printglossary
	% \printglossary[title=List of terms, toctitle=List of terms]

	% bib2gls
	% \printunsrtglossaries % print all types
	% \printunsrtglossary[type={abbreviations},title=List of Abbreviations,style=listgroup]
	% \printunsrtglossary[type={abbreviations},title=List of Abbreviations,style=listhypergroup] % doesn't work
	% \printunsrtglossary[type={symbols},title=List of Symbols,style=listgroup]
	% \printunsrtglossary % main entry

%%%%%%%%%%%%%%%Content%%%%%%%%%%%%%%%
% \mainmatter % separat the number of toc and mainmatter
% % !TEX root = ../notes_template.tex
\chapter*{Preface}
\addcontentsline{toc}{chapter}{Preface}
% \minitoc

% \lipsum % dummy text - remove from real document

\section{Features of this template}
% \epigraph{\emph{... nature isn't classical, dammit, and if you want to make a simulation of nature, you'd better make it quantum mechanical, and by golly it's a wonderful problem, because it doesn't look so easy.}}{Richard Feynman (1981) Simulating physics with computers}
\epigraph{\emph{TeX, stylized within the system as \LaTeX, is a typesetting system which was designed and written by Donald Knuth and first released in 1978. TeX is a popular means of typesetting complex mathematical formulae; it has been noted as one of the most sophisticated digital typographical systems.}}{- \href{https://en.wikipedia.org/wiki/TeX}{Wikipedia}}

\subsection{crossref}
different styles of clickable definitions and theorems
\begin{itemize}
	\item nameref:
		\nameref{def:gaussian_distribution}

	\item autoref:
		\autoref{def:gaussian_distribution},
		\autoref{alg:miller_rabin}

	\item cref:
		\cref{def:gaussian_distribution},

	\item hyperref:
		\hyperref[def:gaussian_distribution]{Gaussian},
\end{itemize}

\subsection{ToC (Table of Content)}
\begin{itemize}
	\item mini toc of sections at the beginning of each chapter
	\item list of theorems, definitions, figures
	\item the chapter titles are bi-directional linked
\end{itemize}

\subsection{header and footer}
fancyhdr
\begin{itemize}
	\item right header: section name and link to the beginning of the section
	\item left header: chapter title and link to the beginning of the chapter
	\item footer: page number linked to ToC of the whole document
\end{itemize}

\subsection{bib}
\begin{itemize}
	\item titles of reference is linked to the publisher webpage e.g., \cite{kitaev2002classical}
	\item backref (go to the page where the reference is cited) e.g., \cite{childsUniversalComputationQuantum2009}
	\item customized video entry in reference like in \cite{babaiGraphIsomorphismQuasipolynomial2016}
\end{itemize}

\subsection{preface, index, quote (epigraph) and appendix}
\myindex{index} page at the end of this document...

\subsection{symbol and glossary (abbreviation)}
examples: 
\gls{real_number},
% \gls{natural_number},
% \gls{complex_number},
\gls{svm},
\gls{v}

\subsubsection{usage}
\begin{itemize}
	\item glossary package 
	\begin{verbatim}
		pdflatex notes_template.tex
		makeglossaries notes_template
		pdflatex notes_template.tex	
	\end{verbatim}

	\item glossary-extra package and bib2gls
	\begin{verbatim}
		pdflatex notes_template.tex
		bib2gls notes_template
		pdflatex notes_template.tex	
	\end{verbatim}
\end{itemize}

\section{Related Tools}
\subsection{VSCode}
Extension: \href{https://marketplace.visualstudio.com/items?itemName=James-Yu.latex-workshop}{Latex Workshop by James Yu}

\subsubsection{settings}

\subsection{lualatex and latexmk}
.latexmkrc configuration file
\begin{verbatim*}
	$pdflatex = 'lualatex -synctex=1 -interaction=nonstopmode --shell-escape %O %S';
	@generated_exts = (@generated_exts, 'synctex.gz');
	$pdf_mode = 1;

	add_cus_dep('glo', 'gls', 0, 'makeglo2gls');
	sub makeglo2gls {
		system("makeindex -s '$_[0]'.ist -t '$_[0]'.glg -o '$_[0]'.gls '$_[0]'.glo");
	}
\end{verbatim*}
To explain ....
\begin{verbatim}
# Also delete the *.glstex files from package glossaries-extra. Problem is,
# that that package generates files of the form "basename-digit.glstex" if
# multiple glossaries are present. Latexmk looks for "basename.glstex" and so
# does not find those. For that purpose, use wildcard.
$clean_ext = "%R-*.glstex";

push @generated_exts, 'glstex', 'glg';

add_cus_dep('aux', 'glstex', 0, 'run_bib2gls');

# PERL subroutine. $_[0] is the argument (filename in this case).
# File from author from here: https://tex.stackexchange.com/a/401979/120853
sub run_bib2gls {
    if ( $silent ) {
    #    my $ret = system "bib2gls --silent --group '$_[0]'"; # Original version
        my $ret = system "bib2gls --silent --group $_[0]"; # Runs in PowerShell
    } else {
    #    my $ret = system "bib2gls --group '$_[0]'"; # Original version
        my $ret = system "bib2gls --group $_[0]"; # Runs in PowerShell
    };

    my ($base, $path) = fileparse( $_[0] );
    if ($path && -e "$base.glstex") {
        rename "$base.glstex", "$path$base.glstex";
    }

    # Analyze log file.
    local *LOG;
    $LOG = "$_[0].glg";
    if (!$ret && -e $LOG) {
        open LOG, "<$LOG";
    while (<LOG>) {
            if (/^Reading (.*\.bib)\s$/) {
        rdb_ensure_file( $rule, $1 );
        }
    }
    close LOG;
    }
    return $ret;
}
\end{verbatim}

\section{Copyright and License}

\begin{itemize}
    \item GitHub Repo: \url{https://github.com/Jue-Xu/Latex-Template-for-Scientific-Style-Book}
    \item Overleaf template: \url{https://www.overleaf.com/latex/templates/latex-template-for-scientific-style-book/ntprxjksmqxx}
\end{itemize}




% \part{Linear Algebra}
% !TEX root = ../notes_template.tex
\chapter{Contents}\label{chp:contents}

% \minitoc

% Temporary course details

\begin{enumerate}
    \item \textbf{Introduction to Rehabilitation Engineering}
    \item \textbf{Mechatronics system design process}
    \item \textbf{Linear Time-Invariant Systems review}
    \item \textbf{Electrical circuits review}
    \begin{itemize}
        \item Electrical cricuit elements: voltage source, current source, resistor, capacitor, inductor.
        \item Electrical power and energy
        \item Kirchoff's laws
        \item Thevenin's and Norton's theorems 
    \end{itemize}
    \item \textbf{Electronics review}
    \begin{itemize}
        \item Diodes, bipolar junction transistors, Field effect transistors
        \item Operational amplifiers
    \end{itemize}
    \item \textbf{Sensors \& Signal conditioning}
    \begin{itemize}
        \item Movements sensors: potentiometer, capactiive sensor, rotary encoder, accelerometer, gyroscope, Hall effect sensor, Tachometer
        \item Force sensor: Straing gauge
        \item Proximity sensor
    \end{itemize}
    \item \textbf{Actuators}
    \begin{itemize}
        \item Solenoids
        \item Brushed DC motor
        \item Models of DC motor
        \item Brushelss DC motors
        \item Stepper motor
    \end{itemize}
    \item \textbf{Micrcontrollers}
    \begin{itemize}
        \item Fundamentals
        \item Interfacing
        \item Communication protocols
        \item Fault detection
    \end{itemize}
    \item \textbf{System dynamics - Bond graph modelling}
    \item \textbf{Automatic Control}
    \begin{itemize}
        \item Feedback systems
        \item Stability analysis
        \item PID Control
        \item Design of feedback control
    \end{itemize}
    \item \textbf{Case Studies}
    \begin{itemize}
        \item Rehabilitation robotics
        \
        \item Functional electrical stimulation
        \item Prosthetic limbs
        \item Mobility aids
        \item Human-machine interaction
    \end{itemize}
\end{enumerate}


% \part{Computer Science}
% % \input{./chapter/complexity.tex}
% % !TEX root = ../notes_template.tex
\chapter{Machine Learning}\label{chp:machine_learning}
\minitoc

\section{Regression}
% \gls{algorithm};
\subsection{Gradient descent}\label{sec:gradient_descent}
\gls{gd};
% \glsxtrshort{gd}

\section{Support Vector Machine}
\gls{svm};
% % \input{./chapter/algorithms.tex}

% \part{Physics}
% % !TEX root = ../notes_template.tex
\chapter{Quantum Mechanics}\label{chp:quantum_mechanics}
\minitoc

\section{Hamiltonian}
\gls{hamiltonian};
% \glsxtrshort{qm};

\section{Path Integral}
\gls{lagrangian}

\section{Quantum Field Theory}
\gls{qft};
% % \input{./chapter/quantum_field_theory.tex}

% \begin{appendices}
% % !TEX root = ../notes_template.tex
\chapter{Formulas}

\section{Gaussian distribution}\label{sec:gaussian_distribution}
\begin{definition}[Gaussian distribution]\label{def:gaussian_distribution}
    \myindex{Gaussian distribution}
\end{definition}

\begin{theorem}[Central limit theorem]\label{thm:central_limit_theorem}
\end{theorem}
% \end{appendices}

\backmatter

%%%%%%%%%%%%%%% Reference %%%%%%%%%%%%%%%

\printbibliography[heading=bibintoc]
\printindex

\end{document}

