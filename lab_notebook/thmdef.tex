%%%%%%%%%%%%%%%%%%%%% Custom-Defintions %%%%%%%%%%%%%%%%%
\declaretheorem[numberwithin=chapter]{theorem}
\declaretheorem[numberwithin=chapter]{axiom}
\declaretheorem[numberwithin=chapter]{lemma}
\declaretheorem[numberwithin=chapter]{proposition}
\declaretheorem[numberwithin=chapter]{claim}
\declaretheorem[numberwithin=chapter]{conjecture}
\declaretheorem[sibling=theorem]{corollary}
\declaretheorem[numberwithin=chapter, style=definition]{definition}
\declaretheorem[numberwithin=chapter, style=definition]{problem}
\declaretheorem[numberwithin=chapter, style=definition]{example}
\declaretheorem[numberwithin=chapter, style=definition]{exercise}
\declaretheorem[numberwithin=chapter, style=definition]{observation}
\declaretheorem[numberwithin=chapter, style=definition]{fact}
\declaretheorem[numberwithin=chapter, style=definition]{construction}
\declaretheorem[numberwithin=chapter, style=definition]{remark}
\declaretheorem[numberwithin=chapter, style=remark]{question}

\usepackage{changepage}
\newenvironment{solution}
    {\renewcommand\qedsymbol{$\square$}\color{blue}\begin{adjustwidth}{0em}{2em}
      \small % Adjust font size to small
      \begin{proof}[\textit Solution.~]}
    {\end{proof}\end{adjustwidth}}

% Define boxed environments for problems and solutions
\newmdenv[
  linecolor=black,
  backgroundcolor=gray!10,
  roundcorner=10pt,
  nobreak=false,
  innertopmargin=2pt,
  innerbottommargin=10pt,
  innerleftmargin=10pt,
  innerrightmargin=10pt,
  linewidth=0.pt
]{boxedstuff}

% Define boxed environments for section summary
\newmdenv[
  linecolor=blue,
  backgroundcolor=blue!5,
  roundcorner=10pt,
  nobreak=false,
  innertopmargin=10pt,
  innerbottommargin=10pt,
  innerleftmargin=10pt,
  innerrightmargin=10pt,
  linewidth=0.pt,
  font=\small
]{secsumbox}

% % Define boxed environments for problems and solutions
% \newtcolorbox{boxedstuff}{
%   colframe=black,
%   colback=gray!10,
%   arc=10pt,
%   top=2pt,
%   bottom=10pt,
%   left=10pt,
%   right=10pt,
%   boxrule=0pt,
%   before skip=10pt,
%   after skip=10pt
% }

% % Define boxed environments for section summary
% \newtcolorbox{secsumbox}{
%   colframe=blue,
%   colback=blue!5,
%   arc=10pt,
%   top=10pt,
%   bottom=10pt,
%   left=10pt,
%   right=10pt,
%   boxrule=0pt,
%   fontupper=\small,
%   before skip=10pt,
%   after skip=10pt
% }

\newcommand{\secsum}[1]{\begin{secsumbox} #1 \end{secsumbox}}
